\documentclass{article}

\usepackage[utf8]{inputenc}
\usepackage[T1]{fontenc}
\usepackage{lmodern}
\usepackage[polish]{babel}
\usepackage[margin=1in]{geometry}

\usepackage{amsmath, amssymb}
\usepackage{amsfonts}
\usepackage{amssymb}
\usepackage{float}

\usepackage{booktabs} % Lepsza jakość tabel
\usepackage{graphicx}
\usepackage{float}

\usepackage{siunitx} 
\sisetup{
    output-exponent-marker = e,
    bracket-numbers = false,
    group-separator = {\,}, 
    scientific-notation = true
}

% Inne
\usepackage{hyperref} 
\hypersetup{
    colorlinks=true,
    linkcolor=blue,
    filecolor=magenta,      
    urlcolor=cyan,
    pdftitle={Obliczenia Naukowe - Lista 2},
    pdfauthor={Karol Wziątek},
}

\title{Sprawozdanie z Laboratorium\\Obliczenia Naukowe - Lista 2}
\author{Karol Wziątek}

\begin{document}
\maketitle
\section{Zadanie 1}
\subsection{Krótki opis problemu}
Powtórzyć zadanie 5 z listy 1, ale usunąć ostatnią 9 z x4 i ostatnią 7 z x5. Sprawdzić, jakie zmiany nastąpią w wyniku dla algorytmów:
\begin{enumerate}
    \item \textbf{w przód} -- $\sum_{i=1}^{n} x_i y_i$
    \item \textbf{w tył} -- $\sum_{i=n}^{1} x_i y_i$
    \item \textbf{sortowanie rosnąco} -- sortujemy iloczyny $x_i y_i$ rosnąco (w zależności od wartości bezwzględnej, osobno dodajemy ujemne i dodatnie)
    \item \textbf{sortowanie malejąco} -- analogicznie jak wyżej, ale sortujemy malejąco
\end{enumerate}
Po modyfikacji x mamy dane:
\[
    \tilde{x} = [2.718281828, -3.141592654, 1.414213562, 0.577215664, 0.301029995]
\]
\[
    y = [1486.2497, 878366.9879, -22.37492, 4773714.647, 0.000185049]
\]
\subsection{Rozwiązanie}
Podejście do problemu jest takie samo jak w zadaniu 5. z listy 1. i znaduje się w pliku zadanie1.jl.

\subsection{Wyniki oraz ich interpretacja}
\begin{table}[h!]
    \centering
    \begin{tabular}{|c|c|c|c|c|}
        \hline
        Alg. & Float32 x * y & Float32 $ \tilde{x} $ * y & Float64 x * y & Float64 $ \tilde{x} $ * y \\
        \hline
        1. & $-0.49994430$ & $-0.49994430$ & $1.025188136829667 \cdot 10^{-10}$ & $-0.004296342739891585$ \\
        \hline
        2. & $-0.45434570$ & $-0.45434570$ & $-1.564330887049437 \cdot 10^{-10}$ & $-0.004296342998713953$\\
        \hline
        3. & $-0.5$ & $-0.5$ & $0$ & $-0.004296342842280865$ \\
        \hline
        4. & $-0.5$ & $-0.5$ & $0$ & $-0.004296342842280865$ \\
        \hline
    \end{tabular}
    \label{tab:iloczyn skalarny}
    \caption{Porównanie wyników dla Float64 i Float32 oraz normalnego i zaburzonego x}
\end{table}
Dla Float32 zaburzony x nie daje innych wyników, z kolei dla Float64 wyniki bardzo się różnią. Prawdziwy wynik to $-1.00657107 \cdot{10}^{-11}$. Błąd wynika z dużej różnicy między rzędami wielkości mnożonych liczb. Float32 jest niewystarczająco precyzyjny, żeby zaburzenie miało na wpływ na rezultat. Inaczej jest w przypadku Float64, który zwraca diamertalnie inny wynik, mimo niewielkiego zaburzenia.
\subsection{Wnioski}
Można wnioskować, że algorytmy wyliczania iloczynu skalarnego są wrażliwe nawet na drobne zmiane, co znaczy, że zadanie jest źle uwarunkowane.

\section{Zadanie 2}
\subsection{Krótki opis problemu}
Wyznaczyć w dwóch programach graficznych wykres funkcji $f(x) = e^{x}\ln{(1 + e^{-x})}$ . Obliczyć granicę tej funkcji dla $\lim_{x \to \infty} f(x)$.
\subsection{Rozwiązanie}
\subsubsection{\texorpdfstring{$\lim_{x \to \infty} f(x)$}{lim x->∞ f(x)}}
Wykorzystując podstawowe przekształcenia i regułe de l'Hopitala, mamy $\lim_{x \to \infty} f(x) = 1$, co zostało potwierdzone przez Wolfram alpha.
\subsubsection{Wykres funkcji}
\begin{figure}[h]
    \centering
    \includegraphics[width=1.0\linewidth,height=2.0\textheight,keepaspectratio]{obraz.png}
    \caption{$f(x)$ GeoGebra}
    \label{fig:geogebra}
\end{figure}

\begin{figure}[h]
    \centering
    \includegraphics[width=1.0\linewidth]{obraz1.png}
    \caption{$f(x)$ Desmos}
    \label{fig:desmos}
\end{figure}
\subsection{Wyniki oraz interpretacja}
Wykresy funkcji nie pokrywają się z jej granicą. Już dla $x = 40$ funkcja przyjmuje wartości, które wydają się być zerem. Wcześniej występują oscylacje, które również nie przedstawiają rzeczywistego wykresu - funkcja powinna być ciągła. Prawdopodobnie dla $x > 40: ln(1 + e^{-x}) = 0$, co wynika z ograniczonych możliwości używanej arytmetyki. Stąd również wartość funkcji wynosi $0$. Osycylacje wynikają z faktu, że występuje duża różnica rzędu wielkości między czynnikami $e^{x}$ oraz $ln(1 + e^{-x})$. Mnożenie takich liczb powoduje duży błąd.
\subsection{Wnioski}
Programy do wyznaczania wykresów również nie są odporne na błędy wynikające z obliczeń zmneinnoprzecinkowych. 

\section{Zadanie 3}
\subsection{Krótki opis problemu}
Rozwiązać układ równań liniowych: $\mathbf{Ax = b}$, dla macierzy $A \in \mathbb{R}^{n \times n}$ i wektora prawych stron $b \in \mathbb{R}^{n}$ metodami:
\begin{itemize}
    \item eleminacji Gaussa, czyli $x = A/b$
    \item wprost, czli $x = A^{-1}b$
\end{itemize}
Eksperymenty należy wykonać dla macierzy Hilberta $H_n$ z rosnącym stopniem oraz dla macierzy losowej $R_n$, gdzie $n = 5, 10, 20$ i rosnącym wskaźnikiem uwarunkowania c = 1, 10, 103, 107, 1012, 1016. Porównwać obliczony $\tilde{x}$ z dokładnym x, czyli policzyć błąd względny.
%Eksperymenty wykonać dla macierzy Hilberta HHHn z rosnącym
%stopniem n > 1 oraz dla macierzy losowej RRRn, n = 5, 10, 20 z rosnącym wskaźnikiem
%uwarunkowania c = 1, 10, 103, 107, 1012, 1016. Porównać obliczony ˜xxx z rozwiązaniem do-
%kładnym xxx = (1, . . . , 1)T ,tj. policzyć błędy względne
\subsection{Rozwiązanie}
Rozwiązanie zadania znajduje się w pliku zadanie3.jl. Zostały tam również dodane metody $hilb(n)$ oraz $matcond(n, c)$, a sam program wyświetla błąd względny dla jednego oraz drugiego algorytmu, wskaźnik uwarunkowania oraz rząd macierzy dla obu rodzajów macierzy.
\subsection{Wyniki i interpretacja}
\begin{table}[h!]
    \centering
    \begin{tabular}{|c|c|c|c|c|}
        \hline
        n & cond(A) & rank(A) & błąd wz. Gauss & błąd wzg. Inwersja \\
        \hline
        2 & 1.928e+01 & 2 & 5.661e-16 & 1.404e-15 \\
        \hline
        3 & 5.241e+02 & 3 & 8.023e-15 & 0.000e+00 \\
        \hline
        4 & 1.551e+04 & 4 & 4.137e-14 & 0.000e+00 \\
        \hline
        5 & 4.766e+05 & 5 & 1.683e-12 & 3.354e-12 \\
        \hline
        6 & 1.495e+07 & 6 & 2.619e-10 & 2.016e-10  \\
        \hline
        7 & 4.754e+08 & 7 & 1.261e-08 & 4.713e-09  \\
        \hline
        8 & 1.526e+10 & 8 & 6.124e-08 & 3.077e-07  \\
        \hline
        9 & 4.932e+11 & 9 & 3.875e-06 & 4.541e-06  \\
        \hline
        10 & 1.602e+13 & 10 & 8.670e-05 & 2.501e-04  \\
        \hline
        11 & 5.223e+14 & 10 & 1.583e-04 & 7.618e-03  \\
        \hline
        12 & 1.752e+16 & 11 & 1.340e-01 & 2.590e-01 \\
        \hline
    \end{tabular}
    \label{tab:macierz hilberta}
    \caption{Porównanie wyników błędów względnych dla różnych metod dla macierzy Hilberta}
\end{table}

\begin{table}[h!]
    \centering
    \begin{tabular}{|c|c|c|c|}
        \hline
        c & rank(A) & błąd wz. Gauss & błąd wzg. Inwersja \\
        \hline
        1.0e+00 & 5 & 1.490e-16 & 1.790e-16 \\
        \hline
        1.0e+01 & 5 & 3.878e-16 & 7.161e-16 \\ 
        \hline
        1.0e+03 & 5 & 1.070e-14 & 1.210e-14 \\  
        \hline
        1.0e+07 & 5 & 1.110e-11 & 9.935e-11 \\ 
        \hline
        1.0e+12 & 5 & 1.249e-05 & 7.716e-06 \\
        \hline
        1.0e+16 & 4 & 1.136e-01 & 1.227e-01 \\
        \hline
    \end{tabular}
    \label{tab:macierz losowa n = 5}
    \caption{Porównanie wyników błędów względnych dla różnych metod dla macierzy losowej dla n = 5}
\end{table}

\begin{table}[H]
    \centering
    \begin{tabular}{|c|c|c|c|}
        \hline
        c & rank(A) & błąd wz. Gauss & błąd wzg. Inwersja \\
        \hline
        1.0e+00 & 10 & 3.511e-16 & 1.955e-16 \\
        \hline
        1.0e+01 & 10 & 3.331e-16 & 4.877e-16 \\
        \hline
        1.0e+03 & 10 & 5.096e-15 & 8.485e-15 \\
        \hline
        1.0e+07 & 10 & 7.364e-11 & 6.421e-11 \\
        \hline
        1.0e+12 & 10 & 5.300e-06 & 6.773e-06 \\
        \hline
        1.0e+16 & 9 & 8.656e-03 & 1.474e-02 \\
        \hline
    \end{tabular}
    \label{tab:macierz losowa n = 10}
    \caption{Porównanie wyników błędów względnych dla różnych metod dla macierzy losowej dla n = 10}
\end{table}

\begin{table}[H]
    \centering
    \begin{tabular}{|c|c|c|c|}
        \hline
        c & rank(A) & błąd wz. Gauss & błąd wzg. Inwersja \\
        \hline
        1.0e+00 & 20 & 3.941e-16 & 4.271e-16 \\
        \hline
        1.0e+01 & 20 & 6.799e-16 & 5.769e-16 \\
        \hline
        1.0e+03 & 20 & 2.617e-14 & 2.728e-14 \\
        \hline
        1.0e+07 & 20 & 1.035e-10 & 1.575e-10 \\ 
        \hline
        1.0e+12 & 20 & 1.762e-05 & 1.254e-05 \\
        \hline
        1.0e+16 & 18 & 1.480e-02 & 8.841e-02 \\
        \hline
    \end{tabular}
    \label{tab:macierz losowa n = 20}
    \caption{Porównanie wyników błędów względnych dla różnych metod dla macierzy losowej dla n = 20}
\end{table}

\begin{itemize}
    \item Dla macierzy Hilberta wraz ze wzrostem wielkości bardzo szybko rośnie zarówno wskaźnik uwarunkowania jak i błąd względy w obu metodach.
    \item Dla macierzy losowych im większy wskaźnik uwarunkowania, tym większy błąd, rozmiar macierzy ani algorytm nie wykazują wpływu na błąd względny.
\end{itemize}


\subsection{Wnioski}
Stąd można wywnioskować, że przy dużym wskaźniku uwarunkowania, tzn. źle uwarunkowanym zadaniu, błąd względny jest duży.

\section{Zadanie 4}
\subsection{Krótki opis problemu}
Dany jest wielomian P w postaci naturalnej oraz w postaci iloczynowej $p = (x-20)(x-19)...(x-1)$. Trzeba znaleźć pierwiastki $z_k$ tego wielomianu, korzystając z postaci naturalnej. Policzyć $ |P(z_k)|$, $|p(z_k)|$ oraz $|z_k-k|$.
\newline
W drugiej części zadania należy powtórzyć eksperyment - zamiast $-210$ ma być $-210 - 2^{-23}$ przy $x^{19}$. 
\subsection{Rozwiązanie}
Rozwiązanie znajduje się w pliku zadania4.jl.
\newline
Na początku przy pomoocy funkcji $roots()$ znajduję pierwiastki dla postaci normalnej i od razu obliczam $|z_k - k|$. Później poprzez podstawienie do $P(x)$ i $p(x)$ wyliczam $|P(z_k)|$ oraz $|p(z_k)|$.
\newline
Analogicznie postępuję dla podpunktu b.

\subsection{Wyniki i interpretacja}

\begin{table}[H]
    \centering
    \begin{tabular}{|c|c|c|c|c|}
        \hline
        k & $z_k$ & $|z_k-k|$ & $|P(z_k)|$ & $|p(z_k)|$  \\
        \hline
        1 & 0.999999999999699 & 0.000000000000301 & 3.569650964788257e+04 & 5.518479490350445e+06 \\
        \hline
        2 & 2.000000000028318 & 0.000000000028318 & 1.762526002666841e+05 & 7.378697629901740e+19 \\
        \hline
        3 & 2.999999999592097 & 0.000000000407903 & 2.791576968824087e+05 & 3.320413931687579e+20 \\
        \hline
        4 & 3.999999983737532 & 0.000000016262468 & 3.027109298899109e+06 & 8.854437035384718e+20 \\
        \hline
        5 & 5.000000665769791 & 0.000000665769791 & 2.291747375656708e+07 & 1.844675205654569e+21 \\
        \hline
        6 & 5.999989245824773 & 0.000010754175227 & 1.290241728420510e+08 & 3.320394888870117e+21 \\
        \hline
        7 & 7.000102002793008 & 0.000102002793008 & 4.805112754602064e+08 & 5.423593016891273e+21 \\
        \hline
        8 & 7.999355829607762 & 0.000644170392238 & 1.637952021896114e+09 & 8.262050140110275e+21 \\
        \hline
        9 & 9.002915294362053 & 0.002915294362053 & 4.877071372550003e+09 & 1.196559421646277e+22 \\
        \hline
        10 & 9.990413042481725 & 0.009586957518275 & 1.363863819545813e+10 & 1.655260133520688e+22 \\
        \hline
        11 & 11.025022932909318 & 0.025022932909318 & 3.585631295130865e+10 & 2.247833297924790e+22 \\
        \hline
        12 & 11.953283253846857 & 0.046716746153143 & 7.533332360358197e+10 & 2.886944688412679e+22 \\
        \hline
        13 & 13.074314032447340 & 0.074314032447340 & 1.960598812433082e+11 & 3.807325552826988e+22 \\
        \hline
        14 & 13.914755591802127 & 0.085244408197873 & 3.575134782310432e+11 & 4.612719853150334e+22 \\
        \hline
        15 & 15.075493799699476 & 0.075493799699476 & 8.216271236455970e+11 & 5.901011420218566e+22 \\
        \hline
        16 & 15.946286716607972 & 0.053713283392028 & 1.551497888049407e+12 & 7.010874106897764e+22 \\
        \hline
        17 & 17.025427146237412 & 0.025427146237412 & 3.694735918486229e+12 & 8.568905825736165e+22 \\
        \hline
        18 & 17.990921352716480 & 0.009078647283520 & 7.650109016515867e+12 & 1.014479936104443e+23 \\
        \hline
        19 & 19.001909818299438 & 0.001909818299438 & 1.143527374972120e+13 & 1.199037620237126e+23 \\
        \hline
        20 & 19.999809291236637 & 0.000190708763363 & 2.792410639368073e+13 & 1.401911741431813e+23 \\
        \hline
    \end{tabular}
    \label{tab:tabela dla wielomianiu Wilkinson'a}
    \caption{Obliczenie pierwiastków postaci normalnej, błędu bezwzględnego, i wartości wielomianu dla postaci normalnej i iloczynowej dla obliczonych wcześniej pierwiastków}
\end{table}




\begin{table}[H]
    \centering
    \begin{tabular}{|c|c|c|c|c|}
        \hline
        k & $z_k$ & $|z_k-k|$ & $|P(z_k)|$ & $|p(z_k)|$  \\
        \hline
        1 & 0.9999999999998357 + 0.0im & 1.6431300764452317e-13 & 2.025987231341821e+04 & 3.013100127684589e+06 \\ \hline
        2 & 2.0000000000550373 + 0.0im & 5.503730804434781e-11 & 3.465414137593836e+05 & 7.378697630296061e+19 \\ \hline
        3 & 2.99999999660342 + 0.0im & 3.3965799062229962e-9 & 2.258059700119701e+06 & 3.320413920110016e+20 \\ \hline
        4 & 4.000000089724362 + 0.0im & 8.972436216225788e-8 & 1.054263179039548e+07 & 8.854437817429642e+20 \\ \hline
        5 & 4.99999857388791 + 0.0im & 1.4261120897529622e-6 & 3.757830916585153e+07 & 1.844672697408419e+21 \\ \hline
        6 & 6.000020476673031 + 0.0im & 2.0476673030955794e-5 & 1.314094332556945e+08 & 3.320450195282313e+21 \\ \hline
        7 & 6.99960207042242 + 0.0im & 0.00039792957757978087 & 3.939355874647618e+08 & 5.422366528916004e+21 \\ \hline
        8 & 8.007772029099446 + 0.0im & 0.007772029099445632 & 1.184986961371896e+09 & 8.289399860984408e+21 \\ \hline
        9 & 8.915816367932559 + 0.0im & 0.0841836320674414 & 2.225522123307771e+09 & 1.160747250177049e+22 \\ \hline
        10 & 10.095455630535774 - 0.6449328236240688im & 0.6519586830380407 & 1.067792123293016e+10 & 1.721289285367071e+22 \\ \hline
        11 & 10.095455630535774 + 0.6449328236240688im & 1.1109180272716561 & 1.067792123293016e+10 & 1.721289285367071e+22 \\ \hline
        12 & 11.793890586174369 - 1.6524771364075785im & 1.665281290598479 & 3.140196234442949e+10 & 2.856840100408096e+22 \\ \hline
        13 & 11.793890586174369 + 1.6524771364075785im & 2.0458202766784277 & 3.140196234442949e+10 & 2.856840100408096e+22 \\ \hline
        14 & 13.992406684487216 - 2.5188244257108443im & 2.518835871190904 & 2.157665405951858e+11 & 4.934647147686795e+22 \\ \hline
        15 & 13.992406684487216 + 2.5188244257108443im & 2.7128805312847097 & 2.157665405951858e+11 & 4.934647147686795e+22 \\ \hline
        16 & 16.73074487979267 - 2.812624896721978im & 2.9060018735375106 & 4.850110893921027e+11 & 8.484694713563005e+22 \\ \hline
        17 & 16.73074487979267 + 2.812624896721978im & 2.825483521349608 & 4.850110893921027e+11 & 8.484694713563005e+22 \\ \hline
        18 & 19.5024423688181 - 1.940331978642903im & 2.4540214463129764 & 4.557199223869993e+12 & 1.318194782060722e+23 \\ \hline
        19 & 19.5024423688181 + 1.940331978642903im & 2.0043294443099486 & 4.557199223869993e+12 & 1.318194782060722e+23 \\ \hline
        20 & 20.84691021519479 + 0.0im & 0.8469102151947894 & 8.756386551865696e+12 & 1.591108408143088e+23 \\ \hline

    \end{tabular}
    \label{tab:tabela dla wielomianiu Wilkinson'a z zaburzeniem}
    \caption{Obliczenie pierwiastków postaci normalnej z zaburzeniem, błędu bezwzględnego, i wartości wielomianu dla postaci normalnej i iloczynowej dla obliczonych wcześniej pierwiastków}
\end{table}

Pierwiastki wielomianu w postaci naturalnej zostały obliczone z dobrą dokładnością (mogłoby się wydawać), ponieważ błąd bezwzględny $< 10^{-1}$. Jednakże przy próbie podstawienia tych pierwiastków do $P(x)$ i $p(x)$ otrzymujemy coraz większe wartości, im większe k. Warto zauważyć, że wartości $|p(z_k)|$ rosną znacznie szybciej niż dla $|P(z_k)|$. W przypadku $P(z_k)$ błąd ten wynika stąd, że współczynniki przy najniższych potęgach $x$ są bardzo duże i wychodzą poza arytmetyke, co generuje duży błąd. Z kolei w przypadku obliczania wartości $p(z_k)$ dochodzi do mnożenia liczb skrajne różnych rzędów, co też prowadzi do dużego błędu.
\newline
W podpunkcie b widać, że nawet lekkie zaburzenie prowadzi do powstania częsci urojnej przy wyliczaniu $z_k$, a błąd bezwględny rośnie nawet do $2$.

\subsection{Wnioski}
Wielomiany o dużych współczynnikach lub wielu pierwiastkach stanowią problem dla maszyny, ponieważ wyliczone wartości z takich wielomianów obraczone są dużym błędem. Ponadto, zadanie jest źle uwarunkowane, skoro mała zamiana w jednym ze współczynników prowadzi do tak dużych rozbieżności w wyniku.

\section{Zadanie 5}

\subsection{Krótki opis problemu}
Należy policzyć kolejne wartości funkcji rekurencyjnej. Zobaczyć różnice między wyliczaniem tych wartości we Float32 i Float64 oraz różnicę między brakiem ingerencji w wartości, a obcięiem kilku znaczących cyfr po otrzymaniu 10. wartości.

\subsection{Rozwiązanie}
Rowiązanie znajduje się w pliku zadanie5.jl.
\newline
Program jest prosty - implementuje tę funkcję rekurencyjną kilkukrotnie tzn. raz dla Float32, raz dla Float64, a raz z obcięciem. 

\subsection{Wyniki i interpretacja}

\begin{table}[H]
    \centering
    \begin{tabular}{|c|c|c|}
        \hline
        n & wartości bez obcięcia & wartości z obcięciem \\
        \hline
        1 & 0.0396999978 & 0.0396999978 \\ \hline
        2 & 0.1540717334 & 0.1540717334 \\ \hline
        3 & 0.5450726151 & 0.5450726151 \\ \hline
        4 & 1.2889779806 & 1.2889779806 \\ \hline
        5 & 0.1715191603 & 0.1715191603 \\ \hline
        6 & 0.5978201628 & 0.5978201628 \\ \hline
        7 & 1.3191138506 & 1.3191138506 \\ \hline
        8 & 0.0562714860 & 0.0562714860 \\ \hline
        9 & 0.2155864984 & 0.2155864984 \\ \hline
        10 & 0.7229133844 & 0.7229133844 \\ \hline
        11 & 1.3238422871 & 1.3241480589 \\ \hline
        12 & 0.0376940966 & 0.0364882238 \\ \hline
        13 & 0.1465138495 & 0.1419587135 \\ \hline
        14 & 0.5216564536 & 0.5073780417 \\ \hline
        15 & 1.2702494860 & 1.2572147846 \\ \hline
        16 & 0.2403967381 & 0.2870922685 \\ \hline
        17 & 0.7882151604 & 0.9011031985 \\ \hline
        18 & 1.2890112400 & 1.1684519053 \\ \hline
        19 & 0.1713951081 & 0.5779681206 \\ \hline
        20 & 0.5974515676 & 1.3097310066 \\ \hline
        21 & 1.3189611435 & 0.0927380323 \\ \hline
        22 & 0.0568690598 & 0.3451511264 \\ \hline
        23 & 0.2177739739 & 1.0232166052 \\ \hline
        24 & 0.7288193703 & 0.9519498348 \\ \hline
        25 & 1.3217444420 & 1.0891739130 \\ \hline
        26 & 0.0459526218 & 0.7977962494 \\ \hline
        27 & 0.1774755567 & 1.2817484140 \\ \hline
        28 & 0.6154094934 & 0.1983565390 \\ \hline
        29 & 1.3254514933 & 0.6753902435 \\ \hline
        30 & 0.0313411988 & 1.3331050873 \\ \hline
        31 & 0.1224179864 & 0.0009130480 \\ \hline
        32 & 0.4447134733 & 0.0036496911 \\ \hline
        33 & 1.1855436563 & 0.0145588033 \\ \hline
        34 & 0.5256333351 & 0.0575993359 \\ \hline
        35 & 1.2736620903 & 0.2204443067 \\ \hline
        36 & 0.2280029058 & 0.7359901071 \\ \hline
        37 & 0.7560556531 & 1.3189160824 \\ \hline
        38 & 1.3093621731 & 0.0570453405 \\ \hline
        39 & 0.0941608250 & 0.2184188515 \\ \hline
        40 & 0.3500445187 & 0.7305550575 \\ \hline
    \end{tabular}
    \label{tab:symulacja rekurencyjna}
    \caption{Porównanie wyników między przeprowadzeniem symulacji bez i z obcięciem cyfr znaczących}
\end{table}

\begin{table}[H]
    \centering
    \begin{tabular}{|c|c|c|}
        \hline
        n & wartości Float32 & wartości Float64 \\
        \hline
        1 & 0.0396999978 & 0.0397000000 \\ \hline
        2 & 0.1540717334 & 0.1540717300 \\ \hline
        3 & 0.5450726151 & 0.5450726260 \\ \hline
        4 & 1.2889779806 & 1.2889780012 \\ \hline
        5 & 0.1715191603 & 0.1715191421 \\ \hline
        6 & 0.5978201628 & 0.5978201201 \\ \hline
        7 & 1.3191138506 & 1.3191137924 \\ \hline
        8 & 0.0562714860 & 0.0562715776 \\ \hline
        9 & 0.2155864984 & 0.2155868392 \\ \hline
        10 & 0.7229133844 & 0.7229143012 \\ \hline
        11 & 1.3238422871 & 1.3238419442 \\ \hline
        12 & 0.0376940966 & 0.0376952973 \\ \hline
        13 & 0.1465138495 & 0.1465183827 \\ \hline
        14 & 0.5216564536 & 0.5216706214 \\ \hline
        15 & 1.2702494860 & 1.2702617739 \\ \hline
        16 & 0.2403967381 & 0.2403521728 \\ \hline
        17 & 0.7882151604 & 0.7881011902 \\ \hline
        18 & 1.2890112400 & 1.2890943028 \\ \hline
        19 & 0.1713951081 & 0.1710848467 \\ \hline
        20 & 0.5974515676 & 0.5965293125 \\ \hline
        21 & 1.3189611435 & 1.3185755880 \\ \hline
        22 & 0.0568690598 & 0.0583776083 \\ \hline
        23 & 0.2177739739 & 0.2232865976 \\ \hline
        24 & 0.7288193703 & 0.7435756764 \\ \hline
        25 & 1.3217444420 & 1.3155883460 \\ \hline
        26 & 0.0459526218 & 0.0700352956 \\ \hline
        27 & 0.1774755567 & 0.2654263545 \\ \hline
        28 & 0.6154094934 & 0.8503519691 \\ \hline
        29 & 1.3254514933 & 1.2321124624 \\ \hline
        30 & 0.0313411988 & 0.3741464896 \\ \hline
        31 & 0.1224179864 & 1.0766291714 \\ \hline
        32 & 0.4447134733 & 0.8291255674 \\ \hline
        33 & 1.1855436563 & 1.2541546501 \\ \hline
        34 & 0.5256333351 & 0.2979069415 \\ \hline
        35 & 1.2736620903 & 0.9253821286 \\ \hline
        36 & 0.2280029058 & 1.1325322627 \\ \hline
        37 & 0.7560556531 & 0.6822410727 \\ \hline
        38 & 1.3093621731 & 1.3326056470 \\ \hline
        39 & 0.0941608250 & 0.0029091569 \\ \hline
        40 & 0.3500445187 & 0.0116112380 \\ \hline
    \end{tabular}
    \label{tab:symulacja rekurencyjna 2}
    \caption{Porównanie wyników między przeprowadzeniem symulacji dla Float32 i Float64}
\end{table}

Widać, że w przypadku obcięcia wartości do $0,722$ kolejne wyniki zaczynają coraz bardziej się różnić. Po kilku wywołaniach (od $n = 19$) wartości wydają się być od siebie niezależne aż do końca eksperymentu. 
\newline
Inaczej wygląda sytuacja przy porównywaniu Float32 i Float64 bez zewnętrznej ingerencji w wartości. Na drugim miejscu po przecinku dochodzi do różnicy dopiero dla $n = 23$, a na pierwszym dla $n = 27$. Wartości wydają się być bliskie sobie aż do samego końca.

\subsection{Wnioski}
Drobna modyfikacja liczby wpływa na zupełni inny "przebieg symulacji". Błąd się skaluje z każdą iteracją. Dodatkowo można by się zastanawiać, który z tych procesów najdokładniej odzwierciedli faktyczny stan populacji przy danych założeniach początkowych. Otóż okazuje się, że żaden, ponieważ każdej iteracji towarzyszy podnoszenie obliczanej wartości do kwadratu. W związku z tym potrzebna jest za każdym razem 2 razy większa precyzja, żeby nie zgubić dokładności, co przy Float32 i Float64 jest niemożliwe. Błąd ten znacznie powiększa się z każdą iteracją. Już po kilku wywołaniach równie dobrze można by generować wartości losowe. Mamy tu doczynienia z \textbf{numeryczną niestabilnością}.


\section{Zadanie 6}

\subsection{Krótki opis problemu}
Należy policzyć kolejne wartości funkcji rekurencyjnej $x_{n+1} = x_{n}^2 + c$. Zaobserwować zachowanie ciągów.

\subsection{Rozwiązanie}
Rowiązanie znajduje się w pliku zadanie6.jl.
\newline
Program jest prosty - implementuje tę funkcję rekurencyjną, dla ułatwienia liczę wartości iteracyjnie.

\subsection{Wyniki i interpretacja}
\begin{table}[H]
    \centering
    \begin{tabular}{|c|c|c|c|}
        \hline
        n & $x_0 = 1$ & $x_0 = 2$ & $x_0 = 1.99999999$ \\
        \hline
         1 & 1.000000000000000 & 2.000000000000000 & 1.999999999999990 \\ \hline
         2 & -1.000000000000000 & 2.000000000000000 & 1.999999999999960 \\ \hline
         3 & -1.000000000000000 & 2.000000000000000 & 1.999999999999840 \\ \hline
         4 & -1.000000000000000 & 2.000000000000000 & 1.999999999999361 \\ \hline
         5 & -1.000000000000000 & 2.000000000000000 & 1.999999999997442 \\ \hline
         6 & -1.000000000000000 & 2.000000000000000 & 1.999999999989768 \\ \hline
         7 & -1.000000000000000 & 2.000000000000000 & 1.999999999959073 \\ \hline
         8 & -1.000000000000000 & 2.000000000000000 & 1.999999999836291 \\ \hline
         9 & -1.000000000000000 & 2.000000000000000 & 1.999999999345164 \\ \hline
        10 & -1.000000000000000 & 2.000000000000000 & 1.999999997380655 \\ \hline
        11 & -1.000000000000000 & 2.000000000000000 & 1.999999989522621 \\ \hline
        12 & -1.000000000000000 & 2.000000000000000 & 1.999999958090484 \\ \hline
        13 & -1.000000000000000 & 2.000000000000000 & 1.999999832361938 \\ \hline
        14 & -1.000000000000000 & 2.000000000000000 & 1.999999329447781 \\ \hline
        15 & -1.000000000000000 & 2.000000000000000 & 1.999997317791575 \\ \hline
        16 & -1.000000000000000 & 2.000000000000000 & 1.999989271173494 \\ \hline
        17 & -1.000000000000000 & 2.000000000000000 & 1.999957084809083 \\ \hline
        18 & -1.000000000000000 & 2.000000000000000 & 1.999828341078044 \\ \hline
        19 & -1.000000000000000 & 2.000000000000000 & 1.999313393778961 \\ \hline
        20 & -1.000000000000000 & 2.000000000000000 & 1.997254046543948 \\ \hline
        21 & -1.000000000000000 & 2.000000000000000 & 1.989023726436175 \\ \hline
        22 & -1.000000000000000 & 2.000000000000000 & 1.956215384326049 \\ \hline
        23 & -1.000000000000000 & 2.000000000000000 & 1.826778629873910 \\ \hline
        24 & -1.000000000000000 & 2.000000000000000 & 1.337120162564000 \\ \hline
        25 & -1.000000000000000 & 2.000000000000000 & -0.212109670864823 \\ \hline
        26 & -1.000000000000000 & 2.000000000000000 & -1.955009487525616 \\ \hline
        27 & -1.000000000000000 & 2.000000000000000 & 1.822062096315173 \\ \hline
        28 & -1.000000000000000 & 2.000000000000000 & 1.319910282828443 \\ \hline
        29 & -1.000000000000000 & 2.000000000000000 & -0.257836845283740 \\ \hline
        30 & -1.000000000000000 & 2.000000000000000 & -1.933520161214129 \\ \hline
        31 & -1.000000000000000 & 2.000000000000000 & 1.738500213821511 \\ \hline
        32 & -1.000000000000000 & 2.000000000000000 & 1.022382993457439 \\ \hline
        33 & -1.000000000000000 & 2.000000000000000 & -0.954733014689007 \\ \hline
        34 & -1.000000000000000 & 2.000000000000000 & -1.088484870662841 \\ \hline
        35 & -1.000000000000000 & 2.000000000000000 & -0.815200686338098 \\ \hline
        36 & -1.000000000000000 & 2.000000000000000 & -1.335447840993894 \\ \hline
        37 & -1.000000000000000 & 2.000000000000000 & -0.216579063984746 \\ \hline
        38 & -1.000000000000000 & 2.000000000000000 & -1.953093509043491 \\ \hline
        39 & -1.000000000000000 & 2.000000000000000 & 1.814574255067817 \\ \hline
        40 & -1.000000000000000 & 2.000000000000000 & 1.292679727154924 \\ \hline
    \end{tabular}
    \label{tab: funkcja rekurencyjna c = -2}
    \caption{Porównanie wyników funkcji rekrencyjnej dla $c = -2$ i stanów początkowych $x_0 = 1, 2, 1.9999999$}
\end{table}

\begin{table}[H]
    \centering
    \begin{tabular}{|c|c|c|c|c|}
        \hline
        n & $x_0 = 1$ & $x_0 = -1$ & $x_0 = 0.75$ & $x_0 = 0.25$ \\
        \hline
          1 & 1.000000000000000 & -1.000000000000000 & 0.750000000000000 & 0.250000000000000 \\ \hline
     2 & 0.000000000000000 & 0.000000000000000 & -0.437500000000000 & -0.937500000000000 \\ \hline
     3 & -1.000000000000000 & -1.000000000000000 & -0.808593750000000 & -0.121093750000000 \\ \hline
     4 & 0.000000000000000 & 0.000000000000000 & -0.346176147460938 & -0.985336303710938 \\ \hline
     5 & -1.000000000000000 & -1.000000000000000 & -0.880162074929103 & -0.029112368589267 \\ \hline
     6 & 0.000000000000000 & 0.000000000000000 & -0.225314721856496 & -0.999152469995123 \\ \hline
     7 & -1.000000000000000 & -1.000000000000000 & -0.949233276114730 & -0.001694341702646 \\ \hline
     8 & 0.000000000000000 & 0.000000000000000 & -0.098956187516497 & -0.999997129206195 \\ \hline
     9 & -1.000000000000000 & -1.000000000000000 & -0.990207672952200 & -0.000005741579369 \\ \hline
    10 & 0.000000000000000 & 0.000000000000000 & -0.019488764426589 & -0.999999999967034 \\ \hline
    11 & -1.000000000000000 & -1.000000000000000 & -0.999620188061125 & -0.000000000065931 \\ \hline
    12 & 0.000000000000000 & 0.000000000000000 & -0.000759479620641 & -1.000000000000000 \\ \hline
    13 & -1.000000000000000 & -1.000000000000000 & -0.999999423190706 & 0.000000000000000 \\ \hline
    14 & 0.000000000000000 & 0.000000000000000 & -0.000001153618256 & -1.000000000000000 \\ \hline
    15 & -1.000000000000000 & -1.000000000000000 & -0.999999999998669 & 0.000000000000000 \\ \hline
    16 & 0.000000000000000 & 0.000000000000000 & -0.000000000002662 & -1.000000000000000 \\ \hline
    17 & -1.000000000000000 & -1.000000000000000 & -1.000000000000000 & 0.000000000000000 \\ \hline
    18 & 0.000000000000000 & 0.000000000000000 & 0.000000000000000 & -1.000000000000000 \\ \hline
    19 & -1.000000000000000 & -1.000000000000000 & -1.000000000000000 & 0.000000000000000 \\ \hline
    20 & 0.000000000000000 & 0.000000000000000 & 0.000000000000000 & -1.000000000000000 \\ \hline
    21 & -1.000000000000000 & -1.000000000000000 & -1.000000000000000 & 0.000000000000000 \\ \hline
    22 & 0.000000000000000 & 0.000000000000000 & 0.000000000000000 & -1.000000000000000 \\ \hline
    23 & -1.000000000000000 & -1.000000000000000 & -1.000000000000000 & 0.000000000000000 \\ \hline
    24 & 0.000000000000000 & 0.000000000000000 & 0.000000000000000 & -1.000000000000000 \\ \hline
    25 & -1.000000000000000 & -1.000000000000000 & -1.000000000000000 & 0.000000000000000 \\ \hline
    26 & 0.000000000000000 & 0.000000000000000 & 0.000000000000000 & -1.000000000000000 \\ \hline
    27 & -1.000000000000000 & -1.000000000000000 & -1.000000000000000 & 0.000000000000000 \\ \hline
    28 & 0.000000000000000 & 0.000000000000000 & 0.000000000000000 & -1.000000000000000 \\ \hline
    29 & -1.000000000000000 & -1.000000000000000 & -1.000000000000000 & 0.000000000000000 \\ \hline
    30 & 0.000000000000000 & 0.000000000000000 & 0.000000000000000 & -1.000000000000000 \\ \hline
    31 & -1.000000000000000 & -1.000000000000000 & -1.000000000000000 & 0.000000000000000 \\ \hline
    32 & 0.000000000000000 & 0.000000000000000 & 0.000000000000000 & -1.000000000000000 \\ \hline
    33 & -1.000000000000000 & -1.000000000000000 & -1.000000000000000 & 0.000000000000000 \\ \hline
    34 & 0.000000000000000 & 0.000000000000000 & 0.000000000000000 & -1.000000000000000 \\ \hline
    35 & -1.000000000000000 & -1.000000000000000 & -1.000000000000000 & 0.000000000000000 \\ \hline
    36 & 0.000000000000000 & 0.000000000000000 & 0.000000000000000 & -1.000000000000000 \\ \hline
    37 & -1.000000000000000 & -1.000000000000000 & -1.000000000000000 & 0.000000000000000 \\ \hline
    38 & 0.000000000000000 & 0.000000000000000 & 0.000000000000000 & -1.000000000000000 \\ \hline
    39 & -1.000000000000000 & -1.000000000000000 & -1.000000000000000 & 0.000000000000000 \\ \hline
    40 & 0.000000000000000 & 0.000000000000000 & 0.000000000000000 & -1.000000000000000 \\ \hline
    \end{tabular}
    \label{tab: funkcja rekurencyjna c = -1}
    \caption{Porównanie wyników funkcji rekrencyjnej dla $c = -1$ i stanów początkowych $x_0 = 1, -1, 0.75, 0.25$}
\end{table}

W przypadku $c = -2$ oraz $x_0 = 1, 2$ widzimy, że wartość już od 2 iteracji pozostaje bez zmian aż do końca, kolejne iteracje nie mają wpływu na zmianę wyniku - \textbf{zachowanie stabilne}. Z kolei dla $c = = -2$ oraz $x_0 = 1.9999999$ widać, że dla każdej kolejnej wartości odchylenie od 2 jest coraz większe i wzrasta też błąd, który wynika z podnoszenia do kwadratu na ograniczonej precyzji - \textbf{zachowanie chaotyczne}

Dla $c = -1 $ obserwujemy zachowanie stabilne dla $x_0 = 1, -1$ i zachowanie niestabilne w pozostałych przypadkach. Ciekawe jest to, że $x_0 = 0.75, 0.25$ po pewnym czasie również zaczynają przyjmować cykliczne wartości $0$ i $-1$, ale to wynika tylko z błędu zaokrąlenia w systemie zmiennopozycyjnym - w rzeczywistości wartości tych funkcji dla takich danych początkowych nigdy nie są całkowite.

\subsection{Wnioski}
Analiza tego typu układów nie jest prosta, ponieważ stany stabilne i niestabilne przeplatają się ze sobą i zależą od warunków początkowych.

\end{document}
